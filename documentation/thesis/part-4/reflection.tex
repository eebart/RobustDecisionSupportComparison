\chapter{Reflection}
\label{chapter-reflection}

\begin{abstract}
    Given the detailed analysis described throughout this thesis and the conclusions reached, this chapter will discuss the implications of these conclusions. The gaps identified in the introduction will be reviewed and addressed in \cref{reflect-gaps}. The contributions made by this study and their added value to both decision makers and analysts is discussed in \cref{reflect-goals}. Finally a discussion of what the two relevant parties, policy analysts and decision makers, can take away from this study is found in \cref{reflect-application}
\end{abstract}

\newpage

\section{Response to Identified Gaps} \label{reflect-gaps}
In the introduction of this thesis, three different gaps in current research were identified that should be addressed with this research. This section will review those gaps and the progress this study has made toward addressing those gaps. For reference, the identified gap is included at the start of each sub-section. 

    \subsection{Research Gap One}
    \textit{Robust optimization techniques have not yet been integrated into an RDM-based decision support process in literature.}
    
    In the review of concepts, this study developed a common RDM-based structure for the three methods considered. Through that effort, traditional robust optimization techniques were integrated into a method identified as multi-objective robust optimization (MORO). The common structure used for MORO and the other two identified methods of robust decision support (shown in \cref{fig:diff-flows-conclusion}) provide a strong foundation for comparisons made in this study by making the differences and similarities between methods very clear. There are, of course, remaining avenues of research with respect to the development of the MORO method, which are discussed in \cref{chapter-futurework}.
    
    \subsection{Research Gap Two}
    \textit{There is a lack of comparative analysis of the identified robust decision support methods that can help decision-makers determine which method is most suitable to their needs.}
    
    As one of the primary goals of this study was to provide an in-depth comparison of the identified methods of robust decision support, this study does provide an as of yet nonexistent comparison of these methods. Unlike most previous comparison literature, which compare different methods of decision support using an application to a single case and policy structure \citep{Gersonius2016,Hall2012,Roach2015}, this study uses three different policy structures for a single problem to develop a comparison, which provides a broader scope for comparing the methods of interest. 
    
    Also unlike several existing comparisons, this study uses a highly stylized policy problem that is not founded in a real-world wicked problem. Using a commonly referenced and highly stylized policy problem reduces the risk of skewed results due to a biased or poorly implemented model. At the same time, however, a simplified problem like the lake problem may not be able to provide enough complexity to fully test the capabilities of the methods under consideration, which may affect the validity of any conclusions made in a comparison using such a problem. The potential shortcomings held by the lake problem are discussed further in \cref{chapter-futurework}. 
    
    \subsection{Research Gap Three}
    \textit{Existing work that compares decision support methods focuses on one specific case and one policy implementation structure, and not the impact of varying policy structures on the results of analysis.}
    
    This study has introduced the concept of comparing decision support methods using three varying policy formulations by comparing three methods of robust decision support against three variations of the problem identified: the shallow lake problem. Two of the structures used have been commonly referenced in policy analysis literature and represent extreme implementations of both static and adaptive policies \citep{Carpenter1999, Quinn2017, Singh2015, Ward2015}. The intertemporal variation represents a static policy where the level of anthropogenic pollution released at every time step is static and defined through 100 independent decision variables. The direct policy search (DPS) variation represents the other extreme - a highly adaptive policy structure where the amount of pollution to be released is determined for every time step based on a formula that is parameterized using 5 decision levers. 
    
    This research proposed a third and more realistic policy implementation structure: planned adaptive DPS. This variation follows the basic structure of the DPS problem but updates the amount of anthropogenic pollution released every 10 time steps, instead of at every time step. This variation produced results that differed from the DPS and intertemporal variations, especially with respect to the robustness of the non-dominated set of alternatives generated with multi-scenario MORDM. Together, these three problem variations support the comparison of decision support methods against multiple policy variations from the same wicked problem. 
    
    Just like the previous two gaps, there is additional research that can be done regarding the development of policy implementation structures that best reflect real-world conditions of policy making. This, as well, is discussed in further detail in \cref{chapter-futurework}.
    
\section{Contemplating Goals of the Study} \label{reflect-goals}
This thesis began by introducing the concept of a wicked problem and the difficulties associated with attempting to solve wicked problems, especially those that include irreversible tipping point behaviors. As these problems are subject to conditions of deep uncertainty, which cannot be easily described, a new type of analysis is required. Several methods have been developed that, instead of attempting to use a single predictive model and seeking a single optimal solution, seek a series of robust policy alternatives which can inform decision making these methods are identified as robust decision support methods for the purposes of this study. 

Despite the existence of several methods of robust decision support, there remains no clear way for decision makers and analysts to compare methods and determine which is most suitable for each individual wicked problems. Previous literature has completed independent and unique comparisons using many different wicked problems and many different focuses for comparison. This leads to the first contribution that this study is making to the area of policy analysis for wicked problems.

    \subsection{A structured comparison framework}
    This study has developed a systematic framework for comparing many decision support methods. Several points of comparison are included and were developed through in-depth analysis of previous literature that has compared policy analysis methods \citep{Gersonius2016,Hall2012,Matrosov2013a,Roach2015}. These previous studies have used a wide variety of comparison metrics, which have led to conclusions that are less generalizable than a systematic approach may offer. 
    
    The comparison framework developed can be easily extended to compare new methods and new wicked problems, as the points of comparison will cover a broad set of differences between methods and problems. There exist points of comparison that are not relevant for the work done for this study, but may be relevant given different methods or models. This study, therefore, sought to develop a broad set of comparison metrics to ensure that it remains applicable for many other decision support methods. Several visualizations were developed to support these comparison metrics that can also be generalized to new methods and models.
    
    Care must be taken when attempting to compare methods of decision support, though. This study made great efforts to ensure that the differences in methods and problems being compared were kept to a minimum by using common definitions and implementations wherever possible. By doing this, any differences observed can be more easily attributed to the single component of each method that is different, and not to the decision support method in general. When considering new methods, it is important to keep in mind which aspects of each method are being compared, and to limit method variation to just those aspects of each method that are being compared. 
    
    For example, this study in the current form examines the impact of different policy alternative evaluation mechanisms in the MOEA-based search for alternatives. Apart from that variation, the remaining structure and implementation of each method is consistent. This allows the analyst to examine the impact of just that element of a robust decision support method. Alternatively, an analyst may want to examine the impact of different MOEAs altogether. If this is the case, the same method should be used with different MOEAs, keeping the remaining elements consistent (including the policy alternative comparison mechanism within each search algorithm), to more clearly examine the impact of different algorithms on the method itself.
    
    Instructions for extending the comparison completed in this thesis using the method, model, and execution implementations developed for this thesis can be found in \cref{appendix-code}. 
        
    \subsection{A comparison of three robust decision support methods}
    The framework for comparison is then used to compare the results from a series of three methods and three model variations. This is the primary goal sought by the main research question. Applying the comparison framework provides information about how each method handles a wicked problem with different policy implementation structures, and more generally demonstrates how the framework can be applied to other methods and models in the future. 
    
    The first significant contribution as a part of this comparison was the formalization of a common structure for all three methods under consideration, as seen in \cref{fig:diff-flows-conclusion}. This is the first time that the MORO method was formalized to follow the RDM structure, as well as the first time that all three methods are formalized in a common way. This structure provides a foundation for implementing each of the three methods consistently and ensuring that the focus of comparison remains on the variable element of the three methods: the comparison mechanism used in an MOEA-based search.  
    
    Given method implementations that follow the RDM-based structures in \cref{fig:diff-flows-conclusion} and model variations that follow the three identified policy structures from \cref{review-structure}, a structured comparison was made. As a significant portion of the three methods use common implementations, several points of comparison in the framework do not result in any significant conclusions about the trade-offs between methods. 
    
    The comparison framework did identify a couple of key differences. First that the more robustness is incorporated into the search process of a method, the more robust the identified policy alternatives will be. But the incorporation of robustness comes with additional computational costs. MORO, which uses robustness the most in the policy alternative determination step, also has a significantly higher computational cost than the other two methods of analysis. 
    
    The framework also revealed inconsistencies in the results for the planned adaptive DPS variation and multi-scenario MORDM method pairing. This pairing was shown to produce an extremely conservative set of policy alternatives, with respect to maintaining low levels of pollution in the lake. And, as utility is a conflicting outcome of interest, the increase in robustness seen for pollution level and reliability is paired with a sharp decrease in robustness with respect to utility for the town. Further analysis showed that the selection of reference scenarios themselves are what led to the conservative set of policy alternatives. Analysis using a set of random reference scenarios did not lead to a significant difference in robustness with respect to MORDM for any model variation. 
    
    It is unclear why an identical mechanism to select a set of reference scenarios led to such different results for the planned adaptive DPS variation as opposed to the intertemporal and DPS variations. This behavior deserves additional study and testing two primary reasons. First, as the planned adaptive DPS lake problem variation is a newly proposed alternative that better mimics real-world policy structures, more detailed study of the behavior of this variation is warranted. Also, as multi-scenario MORDM is a relatively new method, additional study about the impact of different mechanisms for reference scenario selection would be extremely valuable (a deeper discussion of this issue is found in \cref{chapter-futurework}). Together, these two points may shed more light onto the difference in robustness results seen for the planned adaptive DPS variation and multi-scenario MORDM method pairing.
    
    \subsection{A new multi-objective evolutionary search algorithm}
    A side effect of the development and execution of the comparisons found in this study is the development of a new MOEA identified as auto-adaptive $\epsilon$-NSGAII. As described in detail in \cref{hybridnsgaii}, this algorithm combines the best features of traditional $\epsilon$-NSGAII, a generational structure and epsilon dominance, with the strongest elements of Borg: auto-adaptive operator selection and adaptive population sizing. The algorithm was developed in response to the study completed by \citet{Ward2015}, which demonstrated that a search using the intertemporal lake problem fails in all cases except the Borg framework. The auto-adaptive  $\epsilon$-NSGAII algorithm now provides a generational and completely open-sourced alternative to Borg that proved effective in handling the search for all three variations of the lake problem used in this study, including the intertemporal variation as described by \citet{Ward2015}. 
    
    Implementation of this algorithm uses Platypus, a Python-based optimization library that is a translation of the Java-based MOEAFramework package. The implementation itself can be found in the GitHub repository associated with this thesis (see \cref{appendix-code} for more details).

\section{Using the Results}\label{reflect-application}
The contributions of this thesis can be used by both decision makers, also referred to as problem stakeholders, and policy analysts, who work with decision makers to develop models of an identified wicked problem and who provide advice to about potential courses of action and trade-offs within those options. This section discusses potential uses of this study for the two types of people. 

    \subsection{Relevance to Analysts}
    For policy analysts, the comparison framework can help to make an informed selection of which methods to use for an analysis by providing a structured mechanism with which to compare methods under consideration. The framework was developed to account for potential variation of many different aspects of an analysis, so should support consideration of new methods and new policy problems. This framework is most suited to support comparison of methods that generate policy alternatives following the same structure. For example, each of the robust decision support methods is able to search for policies following all three variations identified, and so the results of the different methods can be compared. There exist other decision support methods like dynamic adaptive policy pathways, which provide results of a different form, making it more difficult to compare with other methods when leveraging this comparative framework. 
    
    The specific results from the comparison made across the 9 method and model variation pairings provide analysts with tangible insight into the impact of different comparison mechanisms in a multi-objective search for alternatives. This comparison also provides guidance into how different policy implementation structures will affect the results of the identified methods. 
    
    Finally, the formalization of the MORO method and development of the auto-adaptive $\epsilon$-NSGAII algorithm can lead to stronger analysis of existing and new wicked problems by taking advantage of functionality that already exists. Along with that, the proposed new variation of the lake problem provides a more realistic stylized policy infrastructure which can help improve existing theoretical work that is developing and comparing the many elements of decision making under conditions of deep uncertainty. 
        
    \subsection{Relevance to Decision Makers}
    Decision makers can leverage this thesis to develop a stronger understanding of the impact a method of robust decision support can have on the policy alternatives that are uncovered by the method. The detailed comparison performed across three methods of decision support and three model variations can also give decision makers a stronger understanding about the goals, benefits, and costs of each of the three methods identified. This can help to inform their own method selection, by providing both theoretical and practical analysis of these methods. Care should be taken to generalizing the conclusions made in this study to other wicked policy problems, as these conclusions are founded on a single highly stylized problem and have not been validated using additional deeply uncertain problems. 
    
    Finally, improved theoretical and practical understanding of these methods and the robust decision support process in general can help decision makers and analysts have more productive interactions, leading to stronger models and better decision making, and more effective solutions to the many complex problems facing the world today. 
    
    
