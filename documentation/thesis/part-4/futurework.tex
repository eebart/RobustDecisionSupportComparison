\chapter{Limitations and Future Work}
\label{chapter-futurework}

\begin{abstract}
    This study has involved the development and execution of three different robust decision support methods, the proposal of a new variation of the highly-stylized lake problem, and the comparison of nine model and method pairings. Through this process, several areas that require additional research were identified. This chapter will discuss those areas. Also included in this chapter is a conversation about the limitations of this study. These limitations are, in general, closely related to the areas of future research identified.
\end{abstract}

\newpage

\section{Cases for comparison}
One significant limitation of this study is the fact that only one stylized problem is used for comparison. One of the primary concerns of case-based research and of comparative studies is the ability to generalize conclusions to a broader set of cases, for example to all wicked policy problems with tipping point characteristics, which are of interest to this study. Because this study uses a single case, it is not known how well the conclusions reached will hold given different wicked problems. This limitation ties into a recognized area of future work with respect to this and other policy analysis studies. 

The lake problem is a common and regularly employed stylized case study in studies of policy analysis where there is deep uncertainty \citep{Quinn2017, Ward2015}. The lake problem has been identified as a proxy to a broad class of environmental planning problems \citep{Quinn2017,Singh2015}. However, the lake problem as it exists currently misses key features of many environmental and infrastructure planning problems, such as significant initial investments to avoid tipping points that cannot be undone once the decision is made. The lake problem, instead, uses small decisions every time step (or 10 time steps in the case of the planned adaptive DPS variation) to manage the pollution level, ignoring larger investment options that are often required in planning problems. This type of policy captures the common property of wicked problems in which decision makers have no right to be wrong because decisions are often one-shot operations \citep{Rittel1973}. 

The lake problem also misses a relationship between different policy options that affect the pollution level of the lake. For example, \citet{Kasprzyk2013} discusses the importance of considering both structural investment and non-structural approaches, and the importance of including multiple policy instruments in a planning problem to achieve greater levels of success. By including a single policy measure - the release of pollution into the lake at every time step - analysis that uses the lake problem as a benchmark is unable to represent the impact of managing multiple competing policy instruments. 

Development of additional stylized case studies that incorporates deep uncertainty will provide more established and reliable mechanisms with which to test and compare present and future methods of policy analysis. One possible option for such a stylized problem is the management of fisheries, which also displays tipping point characteristics and may be constructed to include conditions of deep uncertainty. Development of additional highly stylized problems in the policy analysis field would be beneficial not only for this study, but for many other areas of research, as policy analysis of deeply uncertain problems is still a rapidly developing field of research. 

\section{Method-specific limitations}
There are several points of limitation with respect to the implementation details provided for each of the robust decision support methods considered in this study. First, with respect to the MORO method, three points deserve more attention. First is a lack of consideration of the impact that the number of scenarios has on the policy alternative determination phase of MORO. This study specified 50 scenarios for the search process, but no analysis was performed to determine the impact of a smaller number of scenarios on the success of the search. As the number of scenarios has a significant impact on the computational cost of the most expensive robust decision support method, a study of the impact of a smaller number of scenarios on the results of a MORO-based analysis may prove extremely beneficial. Additional attention should be given to the sampling technique used to build the small ensemble of scenarios used in MORO's policy analysis determination phase, and to the effect of using robustness metrics other than domain criterion satisficing on the outcome of the search. The relationship between the number of scenarios used, sampling technique, and robustness metric applied also deserves additional attention. 

Related to the policy alternative determination process of the multi-scenario MORDM process, this study does not include in comparison different mechanisms to select a list of reference scenarios, or a study on the impact of a different number of reference scenarios on the final results of the various problem variations. Both of these elements are worth further study, especially given that multi-scenario MORDM is a newly proposed method to address a recognized shortcoming in traditional MORDM (selecting non-dominated alternatives based on performance in a single reference scenario). 

Also worth additional study is the developed auto-adaptive NSGAII algorithm that was used in the MOEA search for each of the three robust decision support methods. This algorithm is newly proposed and aims to combine benefits of both the NSGAII and Borg algorithms. The auto-adaptive NSGAII algorithm has been compared to traditional $\epsilon$-NSGAII for the intertemporal problem variation, showing more success than $\epsilon$-NSGAII in finding promising sets of policy alternatives. However, the algorithm deserves further study to determine its efficacy in relation to other recognized MOEAs given other policy structures identified in this research. Such an evaluation can follow an existing framework for comparing MOEAs such as was proposed by \citet{Reed2013}, who already assessed the effectiveness, efficiency, reliability, and controlability of several different MOEAs. 

This study also only considers robustness based on one metric. Given that each robustness metric can describe different facets of a policy's robustness and include varying levels of risk aversion, it is worth considering more than just a domain satisficing metric in a comparison to determine whether one method is more or less successful at finding robust policies. 

\section{Model-related limitations}
Finally, further consideration should be given to the proposed policy implementation structure of the planned adaptive DPS variation. Specifically, different lengths of time between updates to the amount of pollution release may be investigated to provide a more thorough exploration of the middle ground between the intertemporal and DPS variations that the planned adaptive variation is attempting to fill. A fourth alternative policy structure may also be considered that mimics the policy implementation structure of the intertemporal version, with fewer static points at which the pollution release amount is updated, instead of a pre-determined update at every time step.
