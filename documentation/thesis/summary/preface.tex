\chapter{Preface}

This study presents the results of 6 months of research completed at TU Delft in partial fulfillment of a masters degree in Engineering and Policy Analysis. It is intended to be read by people with a basic background in exploratory modeling and policy analysis. This study seeks to provide an in depth comparative framework that guides the comparison of multiple methods of decision support, The framework is as applied to a series of three methods under conditions of three variations of a single problem to develop a stronger understanding of how each of the three methods responds to different forms of deeply uncertain problems. 

Anyone seeking insight into the trade-offs that exist between the methods of decision support considered in this study, or who are interested in comparing other methods of decision support is welcome to read this thesis. Included is a thorough background of exploratory modeling, robust decision making a detailed comparative framework, including a development package to replicate these results and run new comparisons, an innovative approach to multi-objective optimization, and a discussion of the trade-offs existing between the selection of methods used in this study.

An early version of this study was presented at the	9th International Congress on Environmental Modelling and Software in June 2018 with a presentation called "On the role of scenarios in designing robust strategies: a comparison of MORDM, multi-scenario MORDM and Robust Optimization".

\vspace{25pt}

This thesis would not have been completed without my friends and family, who provided advice, unending support, and an escape when work stalled and I desperately needed some perspective. 

I would also like to thank my graduation committee (chaired by Prof. Dr. Alexander Verbraeck and with second supervisor Dr. Martijn Warnier) and especially my first supervisor, Dr. Jan Kwakkel for their support and advice throughout the completion of this thesis. The idea for this thesis was first introduced to me by Dr. Kwakkel as an opportunity to explore a wide range of methodologies used in the growing field of decision making for problems with deep uncertainty. Through research into this topic, along with conversations with Dr. Kwakkel, the members of my committee, and other members of the faculty of Technology, Policy, and Management, I was able to gain a deeper understanding of both the power and limitations of decision making involving complex and wicked problems. 

I hope this study is helpful to all who read it, and good luck!

Erin Bartholomew 

28 August 2018
