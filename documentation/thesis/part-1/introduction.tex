\newcounter{gapcounter}
\renewcommand{\thegapcounter}{\stepcounter{gapcounter} \Nwords{gapcounter}}

\chapter{Foundation}
\label{chapter-intro}

In 2000, the United Nations established the Millennium Development Goals \citep{UnitedNations2015}, eight goals that all United Nations members and several international organizations would work to achieve by 2015. In 2015, these goals were renewed and expanded under the name Sustainable Development Goals to a total of 17 objectives \citep{GAR2015}. Included in the list is eliminating poverty and hunger, ensuring clean water and sanitation, developing affordable and clean energy, improving infrastructure, addressing threats brought by climate change, encouraging sustainable living, and more \citep{GAR2015}. 

The manner in which to achieve each sustainable development goal can be considered a wicked problem. Characteristics of wicked problems include no definitive problem formulation, no immediate or ultimate test of a solution, and no single explanation for the cause of the problem or solution to address the problem \citep{Rittel1973}. Wicked problems also frequently demonstrate irreversible tipping point behaviors, where a rapid and significant change in a system can result from small initial changes to that system, and with no way to return to the previous behavior or system state after that tipping point is reached \citep{Gladwell2006, Lenton2013}. Tipping points in policy problems mean that solutions to wicked problems can be one-shot operations with no right for decision makers to be wrong \citep{Rittel1973}. 

Wicked problems, therefore, are subject not only to traditional types of uncertainty that can be addressed by incorporating probabilities and statistics into the analysis, but also to cognitive, strategic, and institutional uncertainties \citep{Bueren2003}. These sources of uncertainty cannot be treated as stochastic functions, but are the result of a lack of information or agreement among decision makers. This is known as deep uncertainty \citep{WalkerLempertKwakkel2013}. Given the deeply uncertain nature of achieving the UN's sustainable development goals, and of solving many other major policy problems, traditional methods of analysis and decision support almost always fall short. The traditional approach follows a predict-then-act structure, which attempts to build a single computational model that incorporates all known information, and uses that model as a surrogate for the real-world system \citep{Bankes1993}. This predictive model is then used in traditional decision analysis to find the optimal solution \citep{Weaver2013}. 

Instead of this traditional predict-then-act approach, methods of analysis have been developed to address the complexities brought by deep uncertainty that fall under the umbrella of exploratory modeling \citep{Bankes1993}. Given that analysis of deeply uncertain problems cannot reliably depend on a single description of the system under consideration, exploratory modeling uses a series of potential explanations, called computational experiments, to analyze a wicked problem and support the decision making process \citep{Bankes2002}. As wicked problems have no one right solution \citep{Rittel1973}, these computational experiments can be used explore the impact of various assumptions made related to identified deeply uncertain factors and to build a set of potential solutions that are perform well over the group of computational experiments, known as robust performance, instead of a single optimal solution \citep{Bankes2002, Kwakkel2016Compare}. 

These methods each provide a different approach to recommending satisfactory solutions to a deeply uncertain policy problem. These methods are identified as robust decision support methods in this study. Given the existence of many different approaches to robust decision support, the question, then, becomes how to determine which method is most appropriate for a specific policy problem. There is an emerging body of literature that compares robust decision support methods \citep{Hall2012, Gersonius2016, Kwakkel2016Compare, Matrosov2013a, Roach2015, Roach2016}. This study will add to that literature by comparing three different variations of Robust Decision Making, a foundational robust decision support method developed by \citet{Lempert2006}. 

\section{Robust Decision Support}
Whereas traditional predict-then-act methods focus on developing a single policy which is designed to maximize utility of the defined predictive model, robust decision support seeks a set of robust policies that achieve satisfactory performance across multiple potential future states of the world \citep{Herman2015, Popper2005, Walker2013}. Robustness can been operationalized in many ways, and each method prioritizes different policy properties, including flexibility, minimizing risk, avoiding regret, or satisficing \citep{McPhail2018}. 

Several methods of robust decision support have been developed, including decision scaling, info-gap, robust decision making, and robust optimization \citep{Herman2015}. The info-gap method focuses on quantifying how far future conditions must deviate from a most likely future state, where larger deviations from that most likely future state are what is used to indicate poor performance \citep{BenHaim2006}. However, when dealing with wicked policy problems that are deeply uncertain, it is extremely unlikely that the identified most likely future state from which to compare will be accurate \citep{Maier2016}. At the same time, decision scaling leverages decision maker feedback to reveal key uncertainties and improve projections made by existing models \citep{Brown2012}. However, decision scaling requires a system that is well-quantified \citep{Brown2012}, and pre-specified alternative policy solutions \citep{Herman2015}, both of which are not possible when addressing policy problems characterized by deep uncertainty. Each of these methods, therefore, require an analysis that incorporates significant assumptions that ignore the key characteristics of a deeply uncertain problem: no one future state of the world can be identified as more likely than another, and that there is fundamental disagreement between decision makers about the desired outcome behavior of a system.

Unlike info-gap and decision scaling, both robust decision making and robust optimization support decision making for policy problems that include deep uncertainty by minimizing the need for a priori assumptions of deeply uncertain factors.

\textit{Note: In this research, the name robust decision support methods is used to refer to the category of methods considered for comparison. This is to avoid confusion with the foundational method described in this study, which is called robust decision making.}

    \subsection{Robust decision making}
    Early efforts to address the complications brought by wicked policy problems and deep uncertainty use ensembles of models and sensitivity analysis to explore a wider spectrum of potential futures, or scenarios, and develop solutions that perform well under that wider spectrum of futures \citep{Bankes1993}. This technique of scenario-focused planning does allow for analysis to highlight the inherent variability in deeply uncertain policy problems. However, it does not include guidance on how to use that new knowledge to rank policy choices and make decisions \citep{Lempert2002}. What has become known as Robust Decision Making (RDM) builds on model-based scenario analysis to evaluate robustness of potential policy solutions over a wide range of plausible futures \citep{Lempert2002}. 
    
    RDM provides a structure for comparing previously identified policy alternatives and for discovering how changes in model properties affect each alternative's performance. That information can then be used to refine the initially identified set of policies to yield a more robust set of alternatives. This structure is iterative and interactive, allowing analysts and decision makers to work together to stress-test and refine potential policies. 
    
    The fact that RDM requires a list of promising policy alternatives from the start can prove a difficult challenge when considering problems with multiple conflicting objectives \citep{Kasprzyk2013}. As the presence of conflicting objectives is a common characteristic of wicked problems \citep{Rittel1973}, methods of decision support that incorporate the consideration of multiple objectives into a formal method are essential to the analysis of wicked problems. Combined with multi-objective evolutionary algorithms (MOEA), which aim to solve many objective problems with four or more objectives, the basic RDM method can be enhanced to support the development of promising policy alternatives despite conflicting objectives. This is known as multi-objective robust decision making \citep{Kasprzyk2013}. 
    
    Recently, multi-objective robust decision making (MORDM) was extended to address a recognized weaknesses: that the analysis must stem from a baseline scenario that is represented with a single set of input data \citep{Watson2017}. Doing so may yield invalid optimizations if the data changes significantly when compared to the original baseline scenario. Multi-scenario MORDM attempts to lower that risk by optimizing under multiple baseline scenarios \citep{Watson2017}.

    \subsection{Robust optimization}
    Different optimization methods may consider a single objective function or multiple objectives. Single objective optimization is rarely sufficient to address policy problems with deep uncertainty, as there are almost always several conflicting objectives that must be considered. Therefore, when supporting the decision making process of deeply uncertain policy problems, multi objective optimization is preferred. Several methods have been developed to support multiple objectives, including the weighted global criterion method, goal programming, Successive Pareto Optimization, and evolutionary algorithms \citep{Coello2006,Marler2004}. Each of the first three methods have significant shortcomings, including generating only one solution at a time or producing invalid results \citep{Coello2006}. Evolutionary algorithms are able to find the set of optimal solutions with one run and aren't affected by the shape of the Pareto front \citep{Coello2006}, and so will be the focus of multi-objective optimization methods that this research considers. 
    
    These traditional multi-objective optimization methods look for the Pareto optimal set of solutions, where each solution in the set is non-dominated by the other members of that set \citep{Deb2006}. However, when considering deep uncertainty, optimal solutions under one potential future can, following another possible path, lead to unacceptable outcomes \citep{Deb2006, McInerney2012}. Several studies have focused on single-objective optimization not for the optimal solution, but for a set of potentially robust solutions \citep{Branke1998, Mulvey1995, Parmee2002, TsutsuiGosh1997}. Ideas from these early approaches were then extended to develop methods for mathematical multi-objective robust optimization (MORO). The MORO method determines robustness by examining how each solution in the discovered Pareto optimal set of solutions responds to changes in key model variables \citep{Deb2006}.
    
    In the context of deep uncertainty, MORO techniques have been used as a small step within existing decision support methods. For example, to configure tipping points of dynamic policies developed using the adaptive robust design method \citep{Hamarat2013} or to determine the most promising sequence of pathways from a larger set defined through dynamic adaptive policy pathways analysis \citep{Kwakkel2015}. multi-objective optimization techniques, however, have not yet been codified into a formal method for decision support. This leads to the first gap that this research will address. 
    
    \begin{researchbox}{Research Gap \thegapcounter}\label{gap-moro}
        Robust optimization techniques have not yet been integrated into an RDM-based decision support process in literature. 
    \end{researchbox}

    \subsection{Comparing MORDM and MORO}\label{gap-comaprativework}
    MORDM has been well established in literature, as have the techniques that MORO requires. Though each method has the common goal of determining a set of policy alternatives that are maximally robust across many potential futures, the path each follows to determine a set of promising policy alternatives differs. MORDM seeks robust solutions by searching for solutions that perform best under uncertainty conditions defined by a base reference scenario and then determining robustness of those alternatives by examining the performance of each promising alternative under a much wider range of uncertainty conditions \citep{Kasprzyk2013}. In contrast, MORO focuses on determining the most robust solutions possible by incorporating robustness into the initial search process through evaluating a potential policy across a small ensemble of uncertainty settings and returning those policies which have the highest robustness for further analysis \citep{HamaratLoonen2014}. There are several articles that compare concepts held by MORDM and MORO to other decision support methods. \citet{Hall2012} and \citet{Matrosov2013b} compare RDM with the Info-Gap methodology and \citet{Roach2015} compares robust optimization with Info-Gap. \citet{Kwakkel2016Compare} compares RDM with a method called Dynamic Adaptive Policy Pathways, which leverages robust optimization techniques to build its recommendations. These articles all compare methods by leveraging real-world case studies. Additional literature compares concepts held by several methods \citep{Dittrich2016, Herman2015, Maier2016}. However, this existing body of work does not yet compare these two methods directly or with respect to policy problems characterized by tipping points, as many wicked problems are. This represents the second gap that this research will address.
    
    \begin{researchbox}{Research Gap \thegapcounter}\label{gap-comparativework}
        There is a lack of comparative analysis of the identified robust decision support methods that can help decision-makers determine which method is most suitable to their needs.
    \end{researchbox}

\section{Policies Developed with Robust Decision Support Methods}\label{gap-policies}
Robust decision support methods have been used to develop policies with different implementation structures, especially with respect to the type of adaptation considered. These methods have each been used to recommend sets of robust policy options that are static and do not change over time \citep{Sozuer2016,Kasprzyk2013}. Each has also been used to develop sets of robust policies that are adaptive, both through automatic responses to adaptation triggers and from manual adjustments at predefined points in time \citep{HamaratLoonen2014, Kwakkel2015, Trindade2017}.

Despite the flexibility of these robust decision support methods to develop policy alternatives following many different implementation structures, the comparative literature that exists and is discussed in \cref{gap-comaprativework} have generally focused on only one policy implementation structure at a time. This is the third and final gap in literature that this research will address.

\begin{researchbox}{Research Gap \thegapcounter}\label{gaptwo}
    Existing work that compares decision support methods focuses on one specific case and one policy implementation formulations, and not the impact of varying policy formulations on the results of analysis.
\end{researchbox}

\section{Thesis Structure}
This thesis will be structured in the following way. First, the remainder of \cref{part-introduce}: Introducing the Problem will establish the goals and methods used in this study (\cref{chapter-research}), and will provide a review of the concepts that are fundamental to answering the identified research question (\cref{chapter-review}). \cref{part-develop}: Design and Development will establish the implementation details for the methods and problem formulations identified in \cref{part-introduce}. Included in this part will also be a description of the points of comparison that will be used in the comparison, which can be found in \cref{dev-comparisons}. Finally, \cref{part-analysis}: Analysis and \cref{part-discussion}: Discussion, will describe the results of analysis, including comparisons as described in \cref{dev-comparisons}) and conclusions that can be made based on these results. \cref{part-discussion} will also answer this study's key questions in \cref{chapter-conclusion}, the implications of those answers for analysts and decision makers in \cref{chapter-reflection} and discuss future avenues of research related to the problem identified and results that were generated in \cref{chapter-futurework}. 
